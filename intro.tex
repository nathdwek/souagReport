Le but de ce projet intégré est de doter un robot d'un système de contrôle à distance. La base mécanique du robot est fournie, et celui-ci est déjà muni d'une première carte à microcontrôleur complète. Cette dernière permet d'une part de s'interfacer avec les moteurs et les encodeurs à partir du premier microcontrôleur, et d'autre part d'accéder à certaines pattes de celui-ci ainsi que des alimentations. Les autres blocs nécessaires doivent être conçus et implémentés par les étudiants et le code des microcontrôleurs doit être produit.

Ce rapport décrit la démarche adoptée, ainsi que les solutions techniques adoptées pour mener à bien ce projet. Dans un premier temps le cahier des charges doit être analysé pour diviser l'objectif final en différent sous-problème bien définis.

\section{Analyse du cahier des charges}
Le robot doit