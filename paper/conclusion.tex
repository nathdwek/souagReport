Dans l’ensemble ce projet nous a permis d’apprendre beaucoup de choses. Nous avons finalement pu appliquer beaucoup de théorie étudiée dans différentes matières et les appliquer à un projet concret. 

Nous avons réussi à respecter le cahier des charges : notre robot est capable de recevoir, traiter, convertir, transmettre et effectuer les différents ordres qu’il reçoit. Les seules difficultés que nous rencontrons encore sont pour les petites distances à parcourir ainsi que les petits angles. Les moteurs n’arrivent pas à être assez précis, leur qualité ne semble pas suffisante pour cela. En cause, la présence de zones mortes assez larges et variables. 

Pour le reste, le robot fonctionne bien, il faut que le volume soit élevé et l’enceinte proche du micro pour qu'il soit correctement perçu, mais le signal est bien interprété par notre chaîne d’acquisition. l’ordre est rapidement transmis à notre chaîne de régulation qui met en marche le robot. Les consignes données sont respectées à un degré tout à fait respectable. 

Pour conclure, nous dirons que, même si ça n'a pas toujours été évident, nous sommes content d'avoir finalement réussi à terminer notre projet et de tout ce que nous avons pu apprendre pour y parvenir.
