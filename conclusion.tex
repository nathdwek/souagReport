Dans l’ensemble ce projet nous a permis d’apprendre beaucoup de choses. Nous avons finalement pu appliquer beaucoup de théorie étudiée dans différentes matières et les appliquer à un projet concret. 

Nous avons réussi à respecter le cahier des charges : notre robot est capable de recevoir, traiter, convertir, transmettre et effectuer les différents ordres qu’il reçoit. Les seules difficultés que nous rencontrons encore sont pour les petites distances à parcourir ainsi que les petits angles. Les moteurs n’arrivent pas à être assez précis, leur qualité ne semble pas suffisante pour cela et nous avons remarqué la présence de zones mortes assez larges et variables qui nous empêchent d’être précis. 

Le reste du robot fonctionne bien, il faut que le volume soit élevé et l’enceinte proche du micro pour être correctement perçu mais le signal est bien interprété par notre chaîne d’acquisition et l’ordre est rapidement transmis à notre chaîne de régulation qui permet la rotation des moteurs. On voit bien notre robot suivre une phase d’accélération ainsi qu’une phase de décélération pendant son fonctionnement. 
