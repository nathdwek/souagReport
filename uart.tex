Ce chapitre couvre la transmission des trames démodulées par le premier microcontrôleur vers le deuxième microcontrôleur. Dans la section suivante, le format des entrées et sorties de ce bloc vont être détaillées.

\section{Première analyse}
Puisqu'on désire limiter le nombre d'opérations effectuées par le microcontrôleur effectuant le traitement du signal audio, les trames de 10 bits renvoyées par la fonction \ilcode{fskDetector} sont envoyées telles quelles au deuxième microcontrôleur. L'entrée du bloc transmission est donc une trame de 10 bits. L'uart, implémenté en hardware des deux côtés est utilisé pour effectuer la transmission. Celui-ci utilise des trames de 8 ou 9 bits, et il faudra donc 2 trames d'uart pour transmettre une trame de FSK. L'émetteur et le récepteur sera donc logiquement des machines à état séquentielles. Plutôt que de simplement reconstituer la trame de 10 bits originale, on choisit que le récepteur renvoie directement d'une part les 2 bits de commande et d'autre part les 8 bits d'arguments contenus dans une transmission.

Les parties récepteur et émetteur vont être abordées en parallèle dans la suite, puisqu'elles sont fortement liées. Tout d'abord, la configuration des modules UART est examinée\footnote{La plupart des paramètres doivent forcément être les mêmes des deux côtés de la transmission pour que celle-ci soit possible.}, ensuite, l'émetteur et le récepteur vont être construits.

\section{Configuration des modules UART}

\begin{minted}[mathescape]{c}
void initUart(void){
    //Config Générale
    U1MODEbits.IREN = 0;//IRDA off.
    U1MODEbits.UEN = 0b00;//Seuls les ports U1TX et U1RX sont utilisés.
                          //=>il ne faut pas config l'hardware flow-control
    U1MODEbits.LPBACK = 0;//0:inter uC. 1: test uC vers lui même.
    U1MODEbits.ABAUD = 0;//Auto Baud off.
    U1MODEbits.BRGH = 1;//16coups de clock par bit envoyé
    //Plus robuste et de toute façon on a un baudrate ridicule.
    U1BRG = BRGVAL;//Fixe le baud rate par la longueur du timer lié
    U1MODEbits.PDSEL = 0b01;//8bit data, bit de parité (paire)
    U1MODEbits.STSEL = 0;//1 stop bit.

    //Modes d'interruption:
    IEC0bits.U1TXIE = 0;//Disable UART TX interrupt
    //=>Osef des deux lignes suivantes:
    //U1STAbits.UTXISEL0 = 0;
    //U1STAbits.UTXISEL1 = 0;//Déclenche une interruption dès qu'il est possible
                           //d'écrire dans le registre d'envoi
    U1STAbits.URXISEL = 0b00;//Déclenche une interruption à chaque trame reçue
    IEC0bits.U1RXIE = 1;//Enable UART RX interrupt

    //Polarité
    U1STAbits.UTXINV = 1;
    U1MODEbits.URXINV = 1;//tout actif à l'état haut

    //Pattes
    TRISBbits.TRISB7 = 1;
    RPINR18bits.U1RXR = 7;//RP7 est en input et U1RX est branché dessus
    RPOR3bits.RP6R = 0b00011;//RP6 est lié à U1TX

    //Start uart et ses composants
    U1MODEbits.UARTEN = 1;//Active l'uart 1
    U1STAbits.UTXEN = 1;//UART prend le controle des ports
}
\end{minted}
